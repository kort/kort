% Glossar
\newglossaryentry{API} {
	name = API,
	description = {Application Programming Interface, Schnittstelle für die Programmierung}
}

\newglossaryentry{OAuth} {
	name = OAuth,
	description = {OAuth ist ein offenes Protokoll, das eine standardisierte, sichere API-Autorisierung erlaubt\cite{oauth}}
}

\newglossaryentry{CRUD} {
	name = CRUD,
	description = {Das Akronym CRUD beschreibt die 4 Standardoperationen einer Datenbank: \textbf{C}reate, \textbf{R}ead, \textbf{U}pdate, \textbf{D}elete\cite{crud}}
}

\newglossaryentry{AntTarget} {
	name = Ant Target,
	description = {Das Buildautomatisierungs-Tool Ant nennt die einzelnen Schritte eines Builds \emph{Target}. Targets sind eine Sammlung von semantisch zusammengehörigen Tasks. Sie können Abhängigkeiten zu anderen Targets aufweisen, welche dann als Vorbedingung zuerst ausgeführt werden\cite{ant-target}}
}

\newglossaryentry{Cloud} {
	name = Cloud,
	description = {Als Cloud oder Cloud-Computing (\emph{Wolke}) bezeichnet man die Gesamtheit aller Dienste, welche ortsunabhängig im Internet angeboten werden. Dies können zum Beispiel Datenspeicher, Server oder Datenbanken oder schlicht Rechenleistung sein. Der grosse Vorteil der Cloud ist, dass sie sehr leicht skalierbar ist, so dass man  die Leistungen dynamisch an den Bedarf angepassen kann\cite{cloud}}
}

\newglossaryentry{HeadlessBrowser} {
	name = headless Browser,
	description = {Bei einem \emph{headless Browser} handelt es sich um einen Browser, welcher ohne grafische Benutzeroberfläche auskommt. Häufig werden sie dazu verwendet, um Serverjobs, welche auf den Browser angewiesen sind, auszuführen}
}

\newglossaryentry{WebApp} {
	name = Web-App,
	description = {Der Begriff Web-App (von der englischen Kurzform für web application), bezeichnet im allgemeinen Sprachgebrauch Apps für mobile Endgeräte wie Smartphones und Tablet-Computer, die über einen in das Betriebssystem integrierten Browser aus dem Internet geladen und so ohne Installation auf dem mobilen Endgerät genutzt werden können\cite{webapp}}
}

\newglossaryentry{REST} {
	name = REST,
	description = {\todo{describe REST}}
}

\newglossaryentry{git} {
	name = git,
	description = {\todo{describe git}}
}

\newglossaryentry{ci} {
	name = Continuous Integration,
	description = {\todo{describe Continuous Integration}}
}


