\chapter*{Inhalt der CD}
% Titel auch in Kopfzeile anzeigen
\markboth{Inhalt der CD}{Inhalt der CD}
% Kapitel in Inhaltsverzeichnis einfügen
\addcontentsline{toc}{chapter}{Inhalt der CD}

\begin{table}[H]
\centering
\begin{tabular}{|p{0.35\twocelltabwidth}|p{0.65\twocelltabwidth}|}
\hline 
\textbf{Pfad} & \textbf{Beschreibung} \\ 
\hline 
\inlinecode{.sencha/} & Konfiguration für Sencha Cmd \\ 
\hline 
\inlinecode{\_DESIGN/} & Grafik-Rohdaten \\ 
\hline 
\inlinecode{\_DOCUMENTATION/} & Dokumentation der Arbeit \\ 
\hline 
\inlinecode{\_DOCUMENTATION/ba-kort-
\newline jhunzike\_soderbol.pdf} & Dokumentation der Arbeit im PDF-Format \\ 
\hline 
\inlinecode{app/} & \textsc{Kort} Frontend \\ 
\hline 
\inlinecode{build/Kort/production/} & \textsc{Kort} Production Build (komprimierte JavaScript Sourcefiles) \\ 
\hline 
\inlinecode{docs/} & Generierte Code-Dokumentationen (\textsc{Kort} Frontend, \textsc{Kort} Backend, Ext.ux.LeafletMap) \\ 
\hline 
\inlinecode{i18n/} & Internationalisierungs-Plugin für Sencha Touch \\ 
\hline 
\inlinecode{jsduck/} & Konfiguration zur Gererierung der JSDuck Code-Dokumentation \\ 
\hline 
\inlinecode{lib/} & Verwendete Library-Pakete \\ 
\hline 
\inlinecode{resources/} & Ressourcen, welche von \textsc{Kort} verwendet werden (CSS, Bilder, Sprach-Property-Files) \\ 
\hline 
\inlinecode{server/} & \textsc{Kort} Backend \\ 
\hline 
\inlinecode{test/} & Tests der Use Cases und Libraries \\ 
\hline 
\inlinecode{touch/} & Sencha Touch 2 Library \\ 
\hline 
\inlinecode{ux/} & Sencha Touch Erweiterungs-Komponenten \\ 
\hline 
\inlinecode{app.js} & Einstiegspunkt des \textsc{Kort} Frontends \\ 
\hline 
\inlinecode{app.json} & Sencha Cmd Konfiguration von \textsc{Kort} \\ 
\hline 
\end{tabular}
\end{table}