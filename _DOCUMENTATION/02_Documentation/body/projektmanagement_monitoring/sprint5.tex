\section{Sprint 5}

\textbf{3. Dezember 2012 bis 21. Dezember 2012}

Alle Informationen zum Sprint 5 sind auch in unserem Wiki zu finden:
\url{http://kort.rdmr.ch/redmine/projects/kort/wiki/Sprint_5}

\subsubsection{Hauptaufgaben / Fokussierung im Sprint}

\begin{itemize}
	\item Fertigstellen der begonnenen Funktionalitäten
	\item Fehlerbehebungen
	\item Production-Build erstellen
	\item Dokumentation
	\begin{itemize}
		\item Code-Dokumentation
		\item Projektdokumentation
		\item A0 Poster erstellen
	\end{itemize}
\end{itemize}

\subsubsection{Ziele}
\begin{itemize}
	\item Abschliessen der begonnenen Funktionalitäten
	\begin{itemize}
		\item Gutschreiben der Koins / Badges
		\item Zurückschreiben der Daten zu OSM
		\item OAuth mit OpenStreetMap
	\end{itemize}
	\item Finalisieren der Dokumentation
\end{itemize}

\subsubsection{Abgabe / Deliverables}

\begin{itemize}
	\item Finale Version der App
	\item A0 Poster / Abstract / Management Summary
	\item Finale Version der Dokumentation
\end{itemize}

\subsubsection{Erledigte Arbeiten}

Im letzten Sprint lag der Fokus bei der Finalisierung der Applikation.
So mussten wir Funktionalitäten, welche noch nicht vollständig implementiert waren, fertigstellen und bestehende Fehler beheben.

An zusätzlichen Funktionen haben wir den Login über OpenStreetMap implementiert, sowie das Setzen des eigenen Benutzernamens auf der Profilseite.

Ebanfalls galt ein grosser Teil der Zeitplanung der Dokumentation des Projektes.

\subsubsection{Probleme}
Wir mussten aus Zeitgründen auf das Zurückschreiben der Korrekturen zu \brand{OpenStreetMap} verzichten.