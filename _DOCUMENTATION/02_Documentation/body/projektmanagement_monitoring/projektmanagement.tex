\chapter{Projektmanagement}
\label{projektmanagement}

Wir verwendeten die agile Projektmethodik \emph{Scrum}\footnote{\url{http://www.scrum.org}} und arbeiteten dabei während 14 Wochen.
Die Dauer eines Sprints haben wir auf 2 Wochen festgelegt.
In den ersten beiden Wochen waren wir mit dem Erstellen der Aufgabenstellung beschäftigt.
Für die restlichen 12 Wochen ergeben sich dadurch 6 Sprints.

\subsubsection{Aufwand pro Person}
Für diese Bachelorarbeit werden 12 ECTS-Punkte vergeben, wobei 1 ECTS-Punkt 30 Stunden Arbeitsaufwand bedeuten.
Dies ergibt 360 Stunden pro Person, was etwas mehr als 3 Arbeitstagen pro Woche entspricht.

\subsubsection{Anpassung der Sprintdauer}
Wir mussten während des Verlaufs des Projektes feststellen, dass die Sprintdauer von 2 Wochen etwas zu kurz ist und haben deshalb die letzten 6 Wochen mit 2 Sprints à 3 Wochen ersetzt.
Insgesamt wurden im Projekt 5 Sprints durchgeführt. 

\subsubsection{Storypoints pro Sprint}
Als Ausgangslage haben wir 12 Storypoints pro Sprint angenommen, wobei 1 Storypoint ungefähr einem Arbeitstag entspricht. 
Dieser Wert hat sich gut bewährt und wir mussten ihn lediglich bei der Anpassungen der Sprintdauer auf 17 Punkte erhöhen.

% Sprint 1
\section{Sprint 1}

\todo[inline]{Text anpassen für Sprint 1}

Alle Informationen zum Sprint 1 sind auch in unserem Wiki zu finden:
\url{http://redmine.rdmr.ch/redmine/projects/gftprototype/wiki/Sprint_1}

\subsubsection{Hauptaufgaben / Fokussierung im Sprint}

\begin{itemize}
	\item Aufsetzen der Infrastruktur (Repository, Projektmanagement, Entwicklungsumgebung, Server)
	\item Einarbeitung in die Thematik
	\begin{itemize}
		\item GIS
		\item Google Fusion Tables (GFT)
		\item allenfalls weitere \gls{API}s
	\end{itemize}
	\item Erarbeitung eines ersten Roundtrips um Daten von und zu Google Fusion Tables zu schicken/empfangen
	\item Projektsetup
	\begin{itemize}
		\item GitHub\footnote{\url{http://www.github.com}} als Repository
		\item LaTex\footnote{\url{http://www.latex-project.org}} für Dokumentation
		\item Jenkins \footnote{\url{http://jenkins-ci.org}} für Continuous Integration (CI)
		\item Redmine \footnote{\url{http://www.redmine.org}} für Projektmanagement/Wiki/Bugtracker
		\item Scrum als Methodik
	\end{itemize}
\end{itemize}

\subsubsection{Ziele}
\begin{itemize}
	\item Google Fusion Table \gls{API} kennen lernen, Potential abschätzen können
	\item Erster Prototyp mit GFT bauen (Roundtrip mit \gls{CRUD}-Operationen)
\end{itemize}

\subsubsection{Abgabe / Deliverables}
Wir sind im ersten Sprint gut vorangekommen und konnten mit zahlreichen aufeiander aufbauenden Beispielen lernen wie das \gls{API} funktioniert und welche Möglichkeiten es bietet: Abfragen erstellen, Zugriff via \gls{API}, Zugriff via Google Maps Layer (FusionTablesLayer).

\begin{itemize}
	\item Repository, Build-Server und Projekmanagement-Tool aufgesetzt (siehe Kapitel \ref{infrastruktur})
	\item Lauffähiger Prototyp mit Unit-Test für \gls{CRUD}-Operationen
	\item Zahlreiche Beispiele um die Funktionsweise des \gls{API}s zu testen
\end{itemize}

\subsubsection{Probleme}
Es ist uns nicht gelungen den Roundtrip zu erstellen, d.h. Daten vom Benutzer in Fusion Tables zu speichern und diese wieder abzufragen. Wie sich herausgestellt hat, müssen schreibende Zugriffe autorisiert sein. Dazu empfiehlt Google \gls{OAuth} zu benutzen. Diese Thematik war schlicht zu gross um im ersten Sprint anzuschauen. Die schreibenden Zugriffe sind deshalb Ziel für den 2. Sprint geworden. 

% Sprint 2
\section{Sprint 2}

\textbf{15. Oktober 2012 bis 28. Oktober 2012}

Alle Informationen zum Sprint 2 sind auch in unserem Wiki zu finden:
\url{http://kort.rdmr.ch/redmine/projects/kort/wiki/Sprint_2}

\subsubsection{Hauptaufgaben / Fokussierung im Sprint}

\begin{itemize}
	\item Alternative für OpenLayers finden
	\item Aufbau der Fehler-Datenbank
	\item Aufsetzen der REST-APIs für DB-Zugriff
	\begin{itemize}
		\item REST-API auf Heroku für Sencha-App
		\item REST-API auf DB-Server für Zugiff von Heroku
	\end{itemize}
	\item Requirements
	\begin{itemize}
		\item User Szenarien
		\item Paper Prototype
	\end{itemize}
\end{itemize}

\subsubsection{Ziele}
\begin{itemize}
	\item Roundtrip von Datenbank $\rightarrow$ Heroku $\rightarrow$ \gls{WebApp} und zurück (Daten lesen und schreiben)
	\item Anforderungen klar machen für Ausrichtung der App
	\begin{itemize}
		\item Was ist möglich?
		\item Was soll tatsächlich abgebildet werden?
	\end{itemize}
	\item Technische Hürden im Middletier überwinden, um sich auf Business-Logik und UI zu konzentrieren
\end{itemize}

\subsubsection{Abgabe / Deliverables}

\begin{itemize}
	\item Lauffähiger Prototyp
	\begin{itemize}
		\item Anzeigen von Bugs
		\item Eintragen von Daten in DB
	\end{itemize}
	\item User Szenarien
	\item Paper Prototyp
\end{itemize}

\subsubsection{Erledigte Arbeiten}
Während des zweiten Sprints haben wir verschiedene Fehler-Datenquellen evaluiert. Wir entschieden uns für den Einsatz der \brand{KeepRight}-Daten, da diese sehr gut unseren Anforderungen entsprechen. Sie werden automatisiert generiert, was es uns ermöglicht, für die verschiedenen Typen, auch automatisiert eine Lösungsmaske anzuzeigen. Zudem handelt es sich nur um Fehler in Tags der OpenStreetMap-Objekte.

Ebenfalls konnten wir, als Alternative zu OpenLayers, eine Sencha Touch Komponente erstellen, welche \brand{Leaflet} verwendet, um OpenStreetMap-Daten anzuzeigen.
Die Library besitzt ein gutes \gls{API} und wurde speziell für die Verwendung im mobilen Umfeld erstellt.

Schlussendlich konnten wir einen kompletten Roundtrip der Fehler-Daten von unserer Datenbank in die App und das Senden der Lösung wieder zurück in die Datenbank realisieren.

\subsubsection{Probleme}
Aus Zeitgründen konnten wir das Zurückschreiben der Daten zu \brand{OpenStreetMap} noch nicht erledigen.

% Sprint 3
\section{Sprint 3}

\textbf{29. Oktober 2012 bis 11. November 2012}

Alle Informationen zum Sprint 3 sind auch in unserem Wiki zu finden:
\url{http://kort.rdmr.ch/redmine/projects/kort/wiki/Sprint_3}

\subsubsection{Hauptaufgaben / Fokussierung im Sprint}

\begin{itemize}
	\item Backend
	\begin{itemize}
		\item Automatisierung der Fehlerdatenbank
		\item OAuth-Login bei Google/Facebook/Twitter etc.
	\end{itemize}
	\item Frontend
	\begin{itemize}
		\item Detail-Seiten für verschiedene Bug-Typen
		\item Profil-Seite
	\end{itemize}
\end{itemize}

\subsubsection{Ziele}
\begin{itemize}
	\item Welche Bug-Typen eignen sich für unsere App?
	\item Bug-Detailmasken implementiert
	\item OAuth Login und Logout
\end{itemize}

\subsubsection{Abgabe / Deliverables}

\begin{itemize}
	\item Lauffähiger Prototyp
	\begin{itemize}
		\item Anzeigen von Bug-Details (mind. 2 verschiedene Typen)
		\item Profile-Seite
		\item Login/Logout
	\end{itemize}
\end{itemize}

\subsubsection{Erledigte Arbeiten}
In diesem Sprint konzentrierten wir uns auf die Handhabung der verschiedenen Fehlertypen.
Wir mussten zuerst herausfinden, welche Typen sich für unsere App eignen.
Für diese mussten wir dann passende Lösungsmasken im Frontend erstellen.

Das zweite grosse Ziel war der Login bzw. Logout über OAuth. Dafür konnten wir ebenfalls eine erste Lösung erstellen.

\subsubsection{Probleme}
Der Benutzer wird derzeit noch nicht in der Datenbank persistiert. Der Login wird momentan noch nicht der Session gespeichert.

% Sprint 4
\section{Sprint 4}

\textbf{12. November 2012 bis 2. Dezember 2012}

Alle Informationen zum Sprint 4 sind auch in unserem Wiki zu finden:
\url{http://kort.rdmr.ch/redmine/projects/kort/wiki/Sprint_4}

\subsubsection{Hauptaufgaben / Fokussierung im Sprint}

\begin{itemize}
	\item Backend
	\begin{itemize}
		\item Datenbank-Setup
		\item OAuth-Handling auf Server (Refresh-Token)
	\end{itemize}
	\item Frontend
	\begin{itemize}
		\item Validations-Maske erstellen mit Fix-Einträgen, welche zu verifizieren sind
		\item Highscore-Masken
		\item Abschluss Bug Detailmasken
	\end{itemize}
\end{itemize}

\subsubsection{Ziele}
\begin{itemize}
	\item Validations-Maske implementiert
	\item Highscore-Maske implementiert
	\item Komplettes Datenbank-Setup
	\item User-Handling (Client, Server, Persistenz, Refresh-Token)
\end{itemize}

\subsubsection{Abgabe / Deliverables}

\begin{itemize}
	\item Lauffähiger Prototyp
	\begin{itemize}
		\item Validations-Maske zeigt zu validierenden Lösungsvorschläge an
		\item Highscore (global) kann angezeigt werden
		\item Datenbank mit vollständigem Schema (bootstrapped)
	\end{itemize}
\end{itemize}

\subsubsection{Erledigte Arbeiten}
Zu Beginn des Sprints mussten wir die Datenbank nach unserem definierten Schema aufbauen.
Dadurch konnten wir mit der Implementation der weiteren Masken der App (Validation, Highscore) starten.
Zusätzlich haben wir begonnen die Benutzeranmeldungen in der Datenbank zu persistieren.
Während dieses Sprints nahmen wir noch am OSM Stammtisch\footnote{\url{http://wiki.openstreetmap.org/wiki/DE:Switzerland:Z\%C3\%BCrich/OSM-Treffen\#36._OSM-Stammtisch}} teil und konnten dort unsere App der \glslink{Mapper}{Mapping}-Community präsentieren.
Wir erhielten spannende und wichtige Hinweise für die weitere Entwicklung unserer App.

\subsubsection{Probleme}
Leider konnten wir die Persistierung der Benutzer noch nicht komplett abschliessen.
Wir werden dies im nächsten Sprint angehen.

% Sprint 5
\subsection{Sprint 5}

\textbf{03. Dezember 2012 bis 21. Dezember 2012}

Alle Informationen zum Sprint 5 sind auch in unserem Wiki zu finden:
\url{http://kort.rdmr.ch/redmine/projects/kort/wiki/Sprint_5}

\subsubsection{Hauptaufgaben / Fokussierung im Sprint}

\begin{itemize}
	\item Backend
	\begin{itemize}
		\item Datenbank-Setup
		\item OAuth-Handling auf Server (Refresh-Token)
	\end{itemize}
	\item Frontend
	\begin{itemize}
		\item Validations-Maske erstellen mit Fix-Einträgen welche zu verifizieren sind
		\item Highscore-Masken
		\item Abschluss Bug Detailmasken
	\end{itemize}
\end{itemize}

\subsubsection{Ziele}
\begin{itemize}
	\item Validations-Maske implementiert
	\item Highscore-Maske implementiert
	\item Komplettes Datenbank-Setup
	\item User-Handling (Client, Server, Persistenz, Refresh-Token)
\end{itemize}

\subsubsection{Abgabe / Deliverables}

\begin{itemize}
	\item Lauffähiger Prototyp
	\begin{itemize}
		\item Validations-Maske zeigt zu verifizierende Bugs an
		\item Highscore (global) kann angezeigt werden
		\item Datenbank mit vollständigem Schema (bootstrapped)
	\end{itemize}
\end{itemize}

\subsubsection{Erledigte Arbeiten}


\subsubsection{Probleme}
