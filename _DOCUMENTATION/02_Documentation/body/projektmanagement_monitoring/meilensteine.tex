\section{Meilensteine}
\label{meilensteine}

Um den Projektverlauf zu beurteilen, haben wir folgende Meilensteine erstellt:

\subsubsection{Meilenstein 1: Aufbereiten von Fehlerdaten}
\tick Geeignete Fehler finden \\
\tick Fehler in Datenbank speichern \\
\tick Fehler filtern \\
\tick Fehler zugänglich machen

\subsubsection{Meilenstein 2: Lesen von Fehlerdaten}

\tick Zugriff auf Datenbank aus App \\
\tick Persistieren der Daten in App

\subsubsection{Meilenstein 3: Kartenanzeige von Fehlern}

\tick Geeignete Kartendarstellung von Fehler finden \\
\tick Fehler auf der Karte anzeigen \\
\tick Location-based Karte

\subsubsection{Meilenstein 4: OAuth}

\tick Login möglich über OAuth \\
\tick User wird in App persistiert \\
\tick Logout möglich \\
\cross Login mit OSM-OAuth

\subsubsection{Meilenstein 5: Schreiben von Fehlerbehebungen}

\tick Input von User speichern in DB \\
\tick Unterscheidung von Fehlertypen

\subsubsection{Meilenstein 6: Verifizieren von Änderungen}

\tick UI zum verifizieren von Fehlerbehebungen \\
\cross Implementation eines Thresholds für Verfikationen

\subsubsection{Meilenstein 7: Daten zu OSM schicken}

\cross Verifikationsdaten einheitlich via API an OSM senden \\
\cross Prüfung ob eine Änderung zulässig ist

\subsubsection{Meilenstein 8: Gamification-Konzepte (Highscore, Leaderboard, Badges, Achievements)}

\cross Erarbeitung eines Gamifications-Konzepts für OSM \\
\tick Geeignete Elemente in App umsetzen (\tick Highscore, \tick Badges)