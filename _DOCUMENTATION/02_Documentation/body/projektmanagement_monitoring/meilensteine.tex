\section{Meilensteine}
\label{meilensteine}

Nachdem wir die Hauptaktivitäten des Projekts abschätzen konnten, haben wir eine Liste von Meilensteinen definiert.
Diese beschreiben konkrete Ziele welche während dieser Arbeit zu erreichen sind.

Wir haben uns jeweils einige Meilensteine für einen Sprint zum Ziel gesetzt und haben über die Wiki-Seite unseren Fortschritt festgehalten.

Somit gab es neben dem rein agilen Sprint-Planning auch noch funktionale Eckpfeiler.

\subsubsection{Meilenstein 1: Aufbereiten von Fehlerdaten}
\tick Geeignete Fehler finden \\
\tick Fehler in Datenbank speichern \\
\tick Fehler filtern \\
\tick Fehler zugänglich machen

\subsubsection{Meilenstein 2: Lesen von Fehlerdaten}

\tick Zugriff auf Datenbank aus App \\
\tick Persistieren der Daten in App

\subsubsection{Meilenstein 3: Kartenanzeige von Fehlern}

\tick Geeignete Darstellung der Fehler auf der Karte finden \\
\tick Fehler auf der Karte anzeigen \\
\tick Location-based Karte

\subsubsection{Meilenstein 4: OAuth}

\tick Login möglich über OAuth \\
\tick User wird in App persistiert \\
\tick Logout möglich \\
\tick Login mit OSM-OAuth

\subsubsection{Meilenstein 5: Schreiben von Fehlerbehebungen}

\tick Eingaben von Benutzer in DB speichern \\
\tick Unterscheidung von Fehlertypen

\subsubsection{Meilenstein 6: Validieren von Änderungen}

\tick UI zum Validieren von Fehlerbehebungen \\
\tick Implementation eines Thresholds für Validierung

\subsubsection{Meilenstein 7: Daten zu OSM schicken}

\cross Fehlerlösungen einheitlich via API an OSM senden \\
\cross Prüfung, ob eine Änderung zulässig ist

\subsubsection{Meilenstein 8: Gamification-Elemente}

\tick Geeignete Gamification-Elemente für OSM finden \\
\tick Elemente in App umsetzen