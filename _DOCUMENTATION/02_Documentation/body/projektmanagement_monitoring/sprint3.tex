\section{Sprint 3}

\textbf{29. Oktober 2012 bis 11. November 2012}

Alle Informationen zum Sprint 3 sind auch in unserem Wiki zu finden:
\url{http://kort.rdmr.ch/redmine/projects/kort/wiki/Sprint_3}

\subsubsection{Hauptaufgaben / Fokussierung im Sprint}

\begin{itemize}
	\item Backend
	\begin{itemize}
		\item Automatisierung der Fehlerdatenbank
		\item OAuth-Login bei Google/Facebook/Twitter etc.
	\end{itemize}
	\item Frontend
	\begin{itemize}
		\item Detail-Seiten für verschiedene Bug-Typen
		\item Profil-Seite
	\end{itemize}
\end{itemize}

\subsubsection{Ziele}
\begin{itemize}
	\item Welche Bug-Typen eignen sich für unsere App?
	\item Bug-Detailmasken implementiert
	\item OAuth Login und Logout
\end{itemize}

\subsubsection{Abgabe / Deliverables}

\begin{itemize}
	\item Lauffähiger Prototyp
	\begin{itemize}
		\item Anzeigen von Bug-Details (mind. 2 verschiedene Typen)
		\item Profile-Seite
		\item Login/Logout
	\end{itemize}
\end{itemize}

\subsubsection{Erledigte Arbeiten}
In diesem Sprint konzentrierten wir uns auf die Handhabung der verschiedenen Fehlertypen.
Wir mussten zuerst herausfinden, welche Typen sich für unsere App eignen.
Für diese mussten wir dann passende Lösungsmasken im Frontend erstellen.

Das zweite grosse Ziel war der Login bzw. Logout über OAuth. Dafür konnten wir ebenfalls eine erste Lösung erstellen.

\subsubsection{Probleme}
Der Benutzer wird derzeit noch nicht in der Datenbank persistiert. Der Login wird momentan noch nicht der Session gespeichert.