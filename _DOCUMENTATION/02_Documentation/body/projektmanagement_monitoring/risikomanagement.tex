\section{Risikomanagement}
\label{risikomanagement}

Für das Projekt wurden folgende Risiken identifiziert:
\subsection{Technische Risiken}

\subsubsection{OpenStreetMap-Daten können nicht in einer Sencha Touch-Applikation angezeigt werden}
\begin{table}[H]
\centering
\begin{tabular}{|p{0.25\twocelltabwidth}|p{0.75\twocelltabwidth}|}
\hline 
\small{\textbf{Auswirkung}} & Es müsste ein alternatives mobiles Framework gefunden werden, welches mit \brand{OpenStreetMap}-Daten umgehen kann. \\
\hline 
\small{\textbf{Wahrscheinlichkeit}} & tief \\
\hline 
\small{\textbf{Massnahme zur Verhinderung}} & Zu Beginn des Projektes muss eine Sencha Touch Prototyp-Applikation implementiert werden, welche \brand{OpenStreetMap}-Daten auf der Karte darstellt. \\
\hline
\end{tabular}
\end{table}

\subsubsection{Keine geeignete Fehlerdatenquelle vorhanden}
\begin{table}[H]
\centering
\begin{tabular}{|p{0.25\twocelltabwidth}|p{0.75\twocelltabwidth}|}
\hline 
\small{\textbf{Auswirkung}} & Es müsste eine Möglichkeit gefunden werden, vorhandene Fehlerdaten so abzuändern, dass sie für den Einsatz in der App verwendet werden können. \\
\hline 
\small{\textbf{Wahrscheinlichkeit}} & mittel \\
\hline 
\small{\textbf{Massnahme zur Verhinderung}} & Es muss eine Evaluation von bestehenden Fehlerdatenquellen durchgeführt werden. \\
\hline
\end{tabular}
\end{table}

\subsubsection{Schwierigkeiten mit OAuth}
\begin{table}[H]
\centering
\begin{tabular}{|p{0.25\twocelltabwidth}|p{0.75\twocelltabwidth}|}
\hline 
\small{\textbf{Auswirkung}} & OAuth ist als Protokoll bekannt, welches Schwierigkeiten bereiten kann.
Falls sich die Anforderungen nicht umsetzen lassen, muss eine alternative Lösung gefunden werden für den Login. \\
\hline 
\small{\textbf{Wahrscheinlichkeit}} & mittel \\
\hline 
\small{\textbf{Massnahme zur Verhinderung}} & Es muss genügend Zeit für OAuth eingeplant werden. \\
\hline
\end{tabular}
\end{table}

\subsection{Fachliche Risiken}

\subsubsection{OpenStreetMap erlaubt keinen allgemeinen Benutzer zum Zurückschreiben der Fehlerbehebungen}
\begin{table}[H]
\centering
\begin{tabular}{|p{0.25\twocelltabwidth}|p{0.75\twocelltabwidth}|}
\hline 
\small{\textbf{Auswirkung}} & Es müsste eine Möglichkeit gefunden werden, die Fehlerbehebungen trotzdem in \brand{OpenStreetMap} einpflegen zu können. \\
\hline 
\small{\textbf{Wahrscheinlichkeit}} & mittel \\
\hline 
\small{\textbf{Massnahme zur Verhinderung}} & \brand{OpenStreetMap}-Community muss von der Idee hinter \kort{} überzeugt werden. \\
\hline
\end{tabular}
\end{table}

\subsubsection{Zu wenig Erfahrung mit Game-Design}
\begin{table}[H]
\centering
\begin{tabular}{|p{0.25\twocelltabwidth}|p{0.75\twocelltabwidth}|}
\hline 
\small{\textbf{Auswirkung}} & Die App hat keinen Game-Charakter oder wird vom Benutzer nicht als solches erkannt. \\
\hline 
\small{\textbf{Wahrscheinlichkeit}} & gross \\
\hline 
\small{\textbf{Massnahme zur Verhinderung}} & Unser Industriepartner hat viel Erfahrung mit der Entwicklung von Games und kann uns mit Ratschlägen unterstützen.
Daneben haben wir versucht, Designer zu involvieren welche uns unterstützen können. \\
\hline
\end{tabular}
\end{table}