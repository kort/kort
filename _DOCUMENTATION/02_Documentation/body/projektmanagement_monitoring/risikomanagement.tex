\section{Risikomanagement}
\label{risikomanagement}

Für das Projekt wurden folgende Risiken identifiziert:
\todo[inline]{Weitere Risiken dokumentieren: Design für Game, Zeitverlust durch OAuth-Probleme}
\subsection{Technische Risiken}

\subsubsection{OpenStreetMap-Daten können nicht in einer Sencha Touch-Applikation angezeigt werden}
\begin{table}[H]
\centering
\begin{tabular}{|p{0.25\twocelltabwidth}|p{0.75\twocelltabwidth}|}
\hline 
\small{\textbf{Auswirkung}} & Es müsste ein alternatives mobile Framework gefunden werden, welches mit OpenStreetMap-Daten umgehen kann. \\
\hline 
\small{\textbf{Wahrscheinlichkeit}} & tief \\
\hline 
\small{\textbf{Massnahme zur Verhinderung}} & Zu Beginn des Projektes muss eine Sencha Touch Prototyp-Applikation implementiert werden, welche OpenStreetMap-Daten auf der Karte darstellt. \\
\hline
\end{tabular}
\end{table}

\subsubsection{Keine geeignete Fehlerdatenquelle vorhanden}
\begin{table}[H]
\centering
\begin{tabular}{|p{0.25\twocelltabwidth}|p{0.75\twocelltabwidth}|}
\hline 
\small{\textbf{Auswirkung}} & Es müsste eine Möglichkeit gefunden werden, vorhandene Fehlerdaten so abzuändern, dass sie für den Einsatz in der App verwendet werden können. \\
\hline 
\small{\textbf{Wahrscheinlichkeit}} & mittel \\
\hline 
\small{\textbf{Massnahme zur Verhinderung}} & Es muss eine Evaluation von bestehenden Fehlerdatenquellen durchgeführt werden. \\
\hline
\end{tabular}
\end{table}

\subsection{Weitere Risiken}

\subsubsection{OpenStreetMap erlaubt keinen allgemeinen Benutzer zum Zurückschreiben der Fehlerbehebungen}
\begin{table}[H]
\centering
\begin{tabular}{|p{0.25\twocelltabwidth}|p{0.75\twocelltabwidth}|}
\hline 
\small{\textbf{Auswirkung}} & Es müsste eine Möglichkeit gefunden werden, die Fehlerbehebungen trotzdem in OpenStreetMap einpflegen zu können \\
\hline 
\small{\textbf{Wahrscheinlichkeit}} & mittel \\
\hline 
\small{\textbf{Massnahme zur Verhinderung}} & OpenStreetMap-Community muss von der Idee hinter \textsc{Kort} überzeugt werden. \\
\hline
\end{tabular}
\end{table}