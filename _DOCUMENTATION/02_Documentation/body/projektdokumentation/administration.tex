\chapter{Administration}
\label{administration}

\section{Hinzufügen von Fehler-Datenquellen}
\label{additional-error-source}
Die Applikation benutzt die View \inlinecode{kort.all\_errors} (siehe Abbildung \ref{image-kort-database-view-all_errors}) als Schnittstelle zu allen Fehlerdaten.
Wenn neue Fehlerquellen an das System angeschlossen werden sollen, dann muss entsprechend diese View angepasst werden.
Die aktuelle View ist auf die Fehlerquelle \brand{KeepRight} (siehe Kapitel \ref{datenquellen-keepright}) angepasst.

Es werden einige Anforderungen an eine neue Fehlerquelle gestellt:
\begin{itemize}
\item Ein Fehler muss sich konkret auf ein \brand{OpenStreetMap}-Objekt beziehen (\gls{Node}, \gls{Way} oder \gls{Relation})
\item Für die Darstellung auf der Karte muss eine einzelne Koordinate angegeben werden können (keine Flächen oder Wege)
\item Zu einem Fehler gibt es eine aussagekräftige Fehlermeldung
\end{itemize}

\begin{figure}[H]
	\centering
	\includegraphics[scale=0.7]{images/uml/kort-database-view-all_errors}
	\caption{Die View kort.all\_errors}
	\label{image-kort-database-view-all_errors}
\end{figure}

Die Fehler einer neuen Quelle müssen auf die bestehenden Fehlertypen gemappt werden.
Falls dies nicht möglich ist oder es sich um neue Arten von Fehlern handelt, müssen zusätzliche Fehlertypen eingeführt werden (siehe Abschnitt \ref{additional-error-type}).

Daneben müssen für alle neuen Fehlertypen Texte erfasst und allenfalls auch internationalisiert werden (siehe Abschnitt \ref{datenquellen-internationalisierung}).

\section{Hinzufügen von Fehlertypen}
\label{additional-error-type}
Um neue Fehlertypen hinzuzufügen müssen sowohl technische als auch fachliche Anpassungen vorgenommen werden.
Wichtig dabei ist es, den Endbenutzer nicht aus den Augen zu verlieren.
Alle aufgeführten Fehler sollten sich mit wenigen Eingaben lösen lassen.
Die bisher implementierten Fehlertypen benötigen alle nur eine einzige Eingabe.
Für neue Typen ist zu prüfen, ob diese sich bereits mit den bestehenden Views abbilden lassen (siehe Kapitel \ref{view-types}).
Ist dies nicht der Fall, so muss eine neue View erstellt werden.

\subsection{Fehlertypen}
In der Tabelle \ref{kort-bug-types-table} sind die in der App bereits implementierten Fehlertypen beschrieben.

\begin{table}[H]
\centering
\begin{tabular}{|p{0.40\twocelltabwidth}|p{0.37\twocelltabwidth}|}
\hline
\textbf{Fehlertyp} & \textbf{Anzahl vorhandener Fehler \newline \small{(Stand 20.12.2012)}} \\
\hline
Typ des Wegs unbekannt & 1'271'632 \\
\hline
Fehlendes Tempolimit & 463'819 \\
\hline
Sprache des Namens unbekannt & 111'565 \\
\hline
Objekt ohne Namen & 92'080 \\
\hline
Strasse ohne Namen & 74'864 \\
\hline
Kultstätte/Kirche ohne Religion & 15'288 \\
\hline
Autobahn ohne Bezeichner & 1'504 \\
\hline
\hline
\textbf{Total} & \textbf{2'030'752} \\
\hline
\end{tabular}
\caption{Fehlertypen in \kort{}}
\label{kort-bug-types-table}
\end{table}

Diese Typen sind in der Datenbanktabelle \inlinecode{kort.error\_type} (siehe Abbildung \ref{image-kort-database-table-error_type}) definiert.
Im Feld \inlinecode{type} dieser Tabelle befindet sich die eindeutige Identifikation eines Typs.
Diese wird beispielsweise für die Anzeige eines passenden Marker-Icons (auf der Karte) und der Verbindung zu einem passenden View-Typ (siehe Abschnitt \ref{view-types}) verwendet.

\begin{figure}[H]
	\centering
	\includegraphics[scale=0.7]{images/uml/kort-database-table-error_type}
	\caption{Die Tabelle kort.error\_type}
	\label{image-kort-database-table-error_type}
\end{figure}

\subsubsection{Erstellen eines neuen Fehlertyps}
\label{create-new-error-type}
Ein neuer Fehlertyp muss in der Tabelle \inlinecode{kort.error\_type} (siehe Abbildung \ref{image-kort-database-table-error_type}) hinzugefügt werden.
Sobald Fehler als \inlinecode{type} den neuen Typs eingetragen haben, wird dieser auch angezeigt.
Jeder Fehlertyp hat ein zugehöriges Marker-Icon, welches beispielsweise auf der Karte angezeigt wird.
Die Bilddatei hat den gleichen Namen wie das Attribut \inlinecode{type} und befindet sich im Ordner \inlinecode{resources/images/marker\_icons/<type>.png}.

\subsection{View-Typen}
\label{view-types}
All diese Typen werden dann wiederum einem spezifischen View-Typen zugeordnet.
Dieser bestimmt, wie das Formular zum Lösen des Fehlers in der Benutzeroberfläche aussieht.

In Tabelle \ref{kort-view-types-table} sind die bereits vorhandenen View-Typen beschrieben.

\begin{table}[H]
\centering
\begin{tabular}{|p{0.12\twocelltabwidth}|p{0.88\twocelltabwidth}|}
\hline
\textbf{Typ} & \textbf{Beschreibung} \\
\hline
\inlinecode{text} & Rendert ein Text-Eingabefeld \\
\hline
\inlinecode{number} & Rendert ein Zahl-Eingabefeld \\
\hline
\inlinecode{select} & Rendert eine Select-Box mit vorgefüllten Werten aus der Datenbank \\
\hline
\end{tabular}
\caption{View-Typen in \kort{}}
\label{kort-view-types-table}
\end{table}

Wird als View-Typ \inlinecode{select} gewählt, müssen in der Tabelle \inlinecode{kort.answer} (siehe Abbildung \ref{image-kort-database-table-answer}) die möglichen Antworten eingetragen werden.
Darin kann der eigentliche Wert des \brand{OpenStreetMap}-Tags und eine passende Bezeichnung hinterlegt werden.
Als \inlinecode{type} muss der jeweilige Typen-Bezeichner gewählt werden.

\begin{figure}[H]
	\centering
	\includegraphics[scale=0.7]{images/uml/kort-database-table-answer}
	\caption{Die Tabelle kort.answer}
	\label{image-kort-database-table-answer}
\end{figure}

\subsubsection{Erstellen eines neuen View-Typen}
Um einen neuen View-Typen zu definieren muss wie in Code-Ausschnitt \ref{code-kort-add_view_type} beschrieben die Unterscheidung in der Klasse \inlinecode{Kort.view.bugmap.fix.Form} mit dem den neuen Typen ergänzt werden.

\lstset{language=JavaScript}
\begin{lstlisting}[caption=Unterscheidung der View-Typen in Kort.view.bugmap.fix.Form, label=code-kort-add_view_type]
createFixField: function(bug) {
	[...]
	
	if(bug.get('view_type') === '<Neuer View-Typ>') {
		fixField = Ext.create('<Neue View-Klasse>', fieldConfig);
	} else {
		...
	}
	
	[...]
}
\end{lstlisting}

\section{Hinzufügen von Auszeichnungen}
\label{kort-additional-badges}
Die bereits vorhandenen Auszeichnungen sind in Tabelle \ref{kort-badges} beschrieben. Um weitere Auszeichnungen hinzuzufügen, muss folgendermassen vorgegangen werden:

\begin{enumerate}
\item Es muss ein neuer Badge in der Tabelle \inlinecode{kort.badge} (siehe Abbildung \ref{image-kort-database-table-badge}) erstellt werden
\item Zusätzlich muss eine Regel für das Gewinnen des Badges in der Methode \inlinecode{updateBadges()}\footnote{\url{http://kort.herokuapp.com/docs/Kort-backend/classes/Webservice.RewardHandler.html\#method_updateBadges}} der Klasse \inlinecode{RewardHandler} hinzugefügt werden.
\item Für das Frontend muss ein Bild erstellt werden, welches dem Namen (Tabellenattribut \inlinecode{name}) des Badges entspricht. Dieses Bild muss in folgendem Pfad gespeichert werden \inlinecode{/resources/images/badges/<badgename>.png}.
\end{enumerate}

\begin{figure}[H]
	\centering
	\includegraphics[scale=0.7]{images/uml/kort-database-table-badge}
	\caption{Die Tabelle kort.badge}
	\label{image-kort-database-table-badge}
\end{figure}

Sind alle Punkte abgearbeitet, ist der Badge im Frontend ersichtlich und kann von den Benutzern gewonnen werden.

\section{Hinzufügen von OAuth Anbietern}
\label{kort-additional-oauth-provider}
Derzeit ist der Login über die \gls{OAuth}-Dienste von \brand{OpenStreetMap} und \brand{Google} möglich. 
Um zukünftig weitere Login-Dienste anbieten zu können, müssen einige Schritte beachtet werden.
Der Dienst muss folgende Anforderungen erfüllen:
\begin{itemize}
\item Unterstützung für OAuth 1.0, 1.0a oder 2.0
\item Offline-Zugang, d.h. es müssen Aktionen ohne den Benutzer möglich sein
\item Der Dienst muss eine Registrierung der Applikation ermöglichen
\end{itemize}

Wenn ein Dienst diese Anforderung erfüllt, muss die Applikation registriert werden.

Im Backend kann eine entsprechende Subklasse von \inlinecode{AbstractOAuthCallback}\footnote{\url{http://kort.herokuapp.com/docs/Kort-backend/classes/OAuth.AbstractOAuthCallback.html}} erstellt werden. 
Eine so erstellte OAuthCallback-Klasse nimmt dann einen Callback des entsprechenden Anbieters entgegen und authentifiziert den Benutzer.
Das zugehörige Callback-Script wird im Order \inlinecode{/server/oauth2callback} gespeichert.

Schlussendlich muss das Frontend noch angepasst werden.
Dazu gehört, dass der neue Dienst in der Konfiguration (siehe Kapitel \ref{frontend-config}) eingetragen werden muss.
Um einen neuen Login-Button anzuzeigen, muss dieser auf dem Login-Panel hinzugefügt werden.
Die zugehörige Logik befindet sich im Login-Controller

Die Dateien befinden sich unter:
\begin{itemize}
\item Login-Panel: \inlinecode{/app/view/overlay/login/Panel.js}.
\item Login-Controller: \inlinecode{/app/controller/Login.js}
\end{itemize}

\begin{figure}[H]
	\centering
	\includegraphics[scale=0.5]{images/screenshots/kort-screenshot-login}
	\caption{Login-Buttons in \kort{}}
\end{figure}

\section{Reset der Applikation}
\label{kort-reset}
Um die Applikation zurückzusetzen, kann die Stored Procedure \inlinecode{reset\_kort()} verwendet werden. 
Die Prozedur löscht alle laufenden Daten aus der Datenbank, so dass wieder der Ursprungszustand hergestellt ist:
\begin{itemize}
\item Die Punkte (\emph{Koins}) aller Benutzer werden auf 0 zurückgesetzt
\item Die Badges aller Benutzer werden entfernt
\item Alle Lösungsvorschläge von Benutzern werden entfernt
\item Alle Überprüfungen von Lösungsvorschlägen werden entfernt
\end{itemize}

Der Aufruf der Prozedur sieht man im Code-Ausschnitt \ref{kort-reset-cmd}.
Die Prozedur liefert \inlinecode{true} wenn alle Aktionen erfolgreich ausgeführt werden konnten, ansonsten \inlinecode{false}.
\begin{lstlisting}[float, caption=Aufruf von reset\_kort(){,} um die Applikation zurückzusetzen, label=kort-reset-cmd]
select reset_kort();

 reset_kort 
------------
 t
(1 row)
\end{lstlisting}

\section{Frontend Konfiguration}
\label{frontend-config}
Die Konfiguration des Frontends befindet sich in der Datei \inlinecode{/app/util/Config.js}.
Darin sind alle Parameter definiert, welche die Applikation für den Betrieb benötigt.

In der \kort{}-Dokumentation sind alle Parameter detailliert beschrieben:

\url{http://kort.herokuapp.com/docs/Kort/#!/api/Kort.util.Config}