\cleardoublepage
% /osm
\subsection{Webservice: OpenStreetMap \emph{/osm}}
Um \brand{OpenStreetMap}-Objekte auf der Karte anzuzeigen, werden über den \inlinecode{/osm}-Webservice die entsprechenden OSM-Daten geladen.
Der Webservice leitet den Request an das \brand{OSM API}\footnote{\url{http://wiki.openstreetmap.org/wiki/API_v0.6}} weiter und sendet das Resultat an die Webapplikation zurück.

\subsubsection{OpenStreetMap Objekt laden}
\begin{table}[H]
\centering
\begin{tabular}{|p{0.15\threecelltabwidth}|p{0.25\threecelltabwidth}|p{0.6\threecelltabwidth}|}
\hline 
\small{\textbf{URL}} & \multicolumn{2}{p{0.85\threecelltabwidth}|}
{
\inlinecode{http://kort.herokuapp.com/server/webservices/osm/<type>/<id>} 
\newline \newline
\inlinecode{<type>} OSM-Objekttyp
\newline
\inlinecode{<id>} ID des OSM-Objekts
} \\ 
\hline 
\small{\textbf{Methode}} & \multicolumn{2}{p{0.85\threecelltabwidth}|}{\inlinecode{GET}} \\ 
\hline 
\small{\textbf{Parameter}} & \multicolumn{2}{p{0.85\threecelltabwidth}|}{-} \\ 
\hline 
\small{\textbf{Antwort}} & \inlinecode{200 OK} & 
Daten konnten erfolgreich geladen werden. \\
\hline 
\small{\textbf{Antworttyp}} & \multicolumn{2}{p{0.85\threecelltabwidth}|}{\inlinecode{XML}} \\
\hline 
\end{tabular} 
\caption{Webservice OpenStreetMap (GET /osm)}
\end{table}

\textbf{Beispiel:}

\inlinecode{GET http://kort.herokuapp.com/server/webservices/osm/node/1658024260}

\textbf{Antwort:}

\lstset{language=XML}
\begin{lstlisting}[style=examples]
<?xml version="1.0" encoding="UTF-8"?>
<osm version="0.6" generator="OpenStreetMap server" copyright="OpenStreetMap and contributors" attribution="http://www.openstreetmap.org/copyright" license="http://opendatacommons.org/licenses/odbl/1-0/">
  <node id="1658024260" version="1" changeset="10861664" lat="47.5114378" lon="8.5443127" user="pfrauenf" uid="479871" visible="true" timestamp="2012-03-03T20:05:48Z">
    <tag k="amenity" v="fast_food"/>
  </node>
</osm>
\end{lstlisting}