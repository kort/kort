% /highscore
\subsection{Webservice: Highscore \emph{/highscore}}
Über den \inlinecode{/highscore}-Webservice können die Benutzer sortiert nach Anzahl \emph{Koins} geladen werden.

\subsubsection{Highscore laden}
\begin{table}[H]
\centering
\begin{tabular}{|p{0.15\threecelltabwidth}|p{0.25\threecelltabwidth}|p{0.6\threecelltabwidth}|}
\hline 
\small{\textbf{URL}} & \multicolumn{2}{p{0.85\threecelltabwidth}|}
{
\inlinecode{http://kort.herokuapp.com/server/webservices/highscore}
} \\ 
\hline 
\small{\textbf{Methode}} & \multicolumn{2}{p{0.85\threecelltabwidth}|}{\inlinecode{GET}} \\ 
\hline 
\small{\textbf{Parameter}} & \multicolumn{2}{p{0.85\threecelltabwidth}|}{\inlinecode{limit} Maximale Anzahl der Benutzer} \\ 
\hline 
\small{\textbf{Antwort}} & \inlinecode{200 OK} & 
Daten konnten erfolgreich geladen werden. \\
\hline 
\small{\textbf{Antworttyp}} & \multicolumn{2}{p{0.85\threecelltabwidth}|}{\inlinecode{JSON}} \\
\hline 
\end{tabular} 
\caption{Webservice Antworten (GET /highscore)}
\end{table}

\textbf{Beispiel:}

\inlinecode{GET http://kort.herokuapp.com/server/webservices/highscore?limit=10}

\textbf{Antwort:}

\lstset{language=JavaScript}
\begin{lstlisting}[style=examples]
{
	"return": [
		{
			"user_id":"3",
			"username":"tschortsch",
			"koin_count":"140",
			"fix_count":"12",
			"vote_count":"4",
			"ranking":"1",
			"you":true
		},
		{ ... }
	]
}
\end{lstlisting}