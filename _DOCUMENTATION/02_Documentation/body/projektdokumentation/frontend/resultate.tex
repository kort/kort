% Resultate
\section{Resultate}

\textsc{Kort} besteht aus fünf verschiedenen Hauptmasken.
Diese sind in der Applikation über die Tabs im unteren Bereich erreichbar.

\subsection{Maske: Aufträge}
In dieser Maske werden dem Benutzer alle noch nicht gelösten Fehler in seiner Umgebung als Markierungen auf der Karte angezeigt.
Die Fehleranzahl ist dabei auf die 25 nächstgelegenen Fehler limitiert.

Beim Klick auf einen Fehler wird der Benutzer gefragt, ob er diesen auch wirklich lösen kann.
Bestätigt er, wird ihm der Fehler im Detail angezeigt.
In dieser Detailansicht wir ihm zudem je nach Fehlertyp ein Text- oder ein Auswahlfeld angezeigt, in welchem er die Antwort eingeben bzw. auswählen kann. 

Zusätzlich hat er die Möglichkeit sich den Fehler nochmals auf der Karte anzuzeigen zu lassen.
Dabei wird dieser nicht nur als Markierung dargestellt sondern als ein Geometrie-Objekt entsprechend den \gls{OpenStreetMap}-Daten.
Eine Strasse wird dabei beispielsweise als Linie dargestellt und ein Gelände als Polygon.

\begin{figure}[H]
\subfigure{\includegraphics[width=0.3\textwidth]{images/screenshots/kort-screenshot-bugmap}}
\hfill
\subfigure{\includegraphics[width=0.3\textwidth]{images/screenshots/kort-screenshot-bugmap_message_box}}
\hfill
\subfigure{\includegraphics[width=0.3\textwidth]{images/screenshots/kort-screenshot-fix}}
\caption{Maske: Aufträge}
\end{figure}

\subsection{Maske: Prüfen}
In der \emph{Prüfen}-Maske werden dem Benutzer die Lösungen angezeigt, welche noch zu überprüfen sind.
Diese sind dabei nach der Anzahl noch nötiger positiver Überprüfungen gruppiert.
So soll erreicht werden, dass Lösungen die schon bald an \gls{OpenStreetMap} zurückgesendet werden können bevorzugt behandelt werden.
In der Liste werden maximal 25 Überprüfungen angezeigt.

Sobald der Benutzer einen Eintrag auswählt, wird ihm der Fehler auf der Karte angezeigt inklusiver der eingetragenen Lösung.
Er kann nun beurteilen, ob diese Lösung korrekt oder falsch ist.

\begin{figure}[H]
\subfigure{\includegraphics[width=0.3\textwidth]{images/screenshots/kort-screenshot-validation}}
\hfill
\subfigure{\includegraphics[width=0.3\textwidth]{images/screenshots/kort-screenshot-vote}}
\hfill
\subfigure{\includegraphics[width=0.3\textwidth]{images/screenshots/kort-screenshot-reward}}
\caption{Maske: Prüfen}
\end{figure}

\subsection{Maske: Highscore}
In der Highscore werden die Benutzer abhängig der Anzahl von ihnen gewonnener Punkte (sog. \emph{koins}) sortiert.
Es werden jeweils die ersten zehn Platzierungen angezeigt.

\subsection{Maske: Profil}
Im Profil findet man eine Zusammenfassung seiner persönlichen Spielaktivitäten.
Man sieht die Gesamtanzahl der gesammelten \emph{koins} und eine Übersicht der gewonnen Auszeichnungen.

Zusätzlich hat man die Möglichkeit die Auszeichnungen im Grossformat anzuzeigen.

\subsection{Maske: Über Kort}
Auf der \emph{Über Kort}-Seite werden allgemeine Informationen zur Applikation angezeigt.

\begin{figure}[H]
\subfigure{\includegraphics[width=0.3\textwidth]{images/screenshots/kort-screenshot-highscore}}
\hfill
\subfigure{\includegraphics[width=0.3\textwidth]{images/screenshots/kort-screenshot-profile}}
\hfill
\subfigure{\includegraphics[width=0.3\textwidth]{images/screenshots/kort-screenshot-about}}
\caption{Masken: Highscore / Profil / Über Kort}
\end{figure}