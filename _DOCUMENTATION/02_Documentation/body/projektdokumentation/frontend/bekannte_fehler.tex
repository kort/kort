% Bekannte Fehler
\section{Bekannte Fehler}

\subsection{iOS6 - Add to homescreen}
Eine sehr nützliche Funktionalität, welche Apple im Mobile Safari-Browser anbietet ist die \emph{Add to homescreen}-Funktion.

\begin{figure}[H]
\subfigure{\includegraphics[width=0.23\textwidth]{images/bugs/kort-add_to_homescreen_1}}
\hfill
\subfigure{\includegraphics[width=0.23\textwidth]{images/bugs/kort-add_to_homescreen_2}}
\hfill
\subfigure{\includegraphics[width=0.23\textwidth]{images/bugs/kort-add_to_homescreen_3}}
\hfill
\subfigure{\includegraphics[width=0.23\textwidth]{images/bugs/kort-add_to_homescreen_4}}
\caption{iOS Add to homescreen-Funktion}
\end{figure}

Dadurch wird ein App-ähnliches Bookmark der aktuellen Webseite auf dem Homescreen erstellt.
Dieser erhält ein hinterlegtes Icon und einen Titel.
Beim Starten der App erscheint ein Splashscreen, welchen man ebenfalls in der Webseite definieren kann.
Zudem öffnet sich der Browser ohne jegliche Toolbars wie  Adress- oder Navigationsleiste.

Leider befindet sich in iOS6 ein Bug, welcher den Zugriff auf die Geolocation verhindert, wenn die \gls{WebApp} vom Homescreen aus gestartet wird.
Auf StackOverflow\footnote{\url{http://stackoverflow.com/questions/12503815/ios-6-breaks-geolocation-in-webapps-apple-mobile-web-app-capable}} wird der Bug genauer beschrieben.

\subsubsection{Workaround}
Im Sencha Forum\footnote{\url{http://www.sencha.com/forum/showthread.php?246317-2.1.0-RC1-Save-to-home-screen-Geolocation-not-working}} wird als Workaround vorgeschlagen, die Generierung des \newline \inlinecode{apple-mobile-web-app-capable} Metatags in der Sencha Touch Library zu deaktivieren (siehe Code-Ausschnitte \ref{code-ios6-homescreen-metatag} und \ref{code-ios6-bug-workaround}).

\lstset{language=HTML}
\begin{lstlisting}[caption=Metatag{,} welcher iOS6 Bug hervorruft, label=code-ios6-homescreen-metatag]
<meta content="yes" name="apple-mobile-web-app-capable" />
\end{lstlisting}

\lstset{language=JavaScript}
\begin{lstlisting}[caption=iOS6 Bug Workaround, label=code-ios6-bug-workaround]
//meta('apple-mobile-web-app-capable', 'yes');
\end{lstlisting}

Das Deaktivieren dieser Zeile hat aber zur Folge, dass der native Rahmen des Browsers (Adressleiste, Navigationsleiste) wieder angezeigt wird.
Dies ist zwar unschön, löst aber das Problem mit dem Zugriff auf die Geolocation.

\subsection{App Build}
Wie in Abschnitt \ref{sencha-cmd} beschrieben, basiert die App vollständig auf dem Sencha-eigenen Build-Tool \brand{Sencha Cmd}. Darin sind aber noch einige Bugs vorhanden.

Bei \kort{} besteht dabei ein Problem bei der fest eingebauten Komprimierung der JavaScript-Sourcen.
Während diesem Prozess werden lokale Variablennamen mit einzelnen Buchstaben abgekürzt, um die Dateigrösse zu reduzieren.
Dabei treten Konflikte mit der \brand{Leaflet}-Library auf, welche den Buchstaben \emph{L} als Namespace verwendet.

\subsubsection{Workaround}
Um diese Problem zu umgehen, mussten wir das Build-Skript von \brand{Sencha Cmd} minimal anpassen.
So mussten wir die Zeile, welche den \gls{Microloader} komprimiert, auskommentieren (siehe Code-Ausschnitt \ref{senchacmd-workaround}).
Diese befindet sich in folgender Datei:

\inlinecode{/<Sencha Cmd Verzeichnis>/plugins/touch/current/app-build.js} Zeile 362

\lstset{language=JavaScript}
\begin{lstlisting}[caption=Sencha Cmd Workaround, label=senchacmd-workaround]
processIndex = function () {
	[...]
	compressor = new ClosureCompressor();
	microloader = (environment == 'production'
		? 'production'
		: 'testing') +
		'.js';
	_logger.debug("using microloader : {}", microloader);
	content = readFileContent(joinPath(sdk, "microloader", microloader));
	//content = compressor.compress(content);
	remotes = [
		'<script type="text/javascript">' +
			content + ';Ext.blink(' +
			(environment == 'production' ? jsonEncode({
				id:config.id
			}) : appJson) + ')' +
			'</script>'
	];
	[...]
};
\end{lstlisting}