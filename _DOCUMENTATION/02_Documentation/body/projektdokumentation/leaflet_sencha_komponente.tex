\chapter{OpenStreetMap-Daten in Sencha Touch}
\label{leaflet-sencha-komponente}

\section{Sencha Touch Map-Komponente}

Das Sencha Touch 2-Framework bietet zur Darstellung einer Karte lediglich eine Google Maps-Komponente an.
Diese ist stark auf das Google Maps API ausgerichtet und kann deshalb nicht für andere Kartendaten verwendet werden.

Um trotzdem Daten von \brand{OpenStreetMap} verwenden zu können, mussten wir eine neue Sencha Touch Map-Komponente erstellen, welche eine für diesen Zweck vorgesehene Library verwendet.

\section{Leaflet Map-Komponente}

Zur Darstellung der OSM-Daten verwendeten wir zuerst die \brand{OpenLayers}\footnote{\url{http://openlayers.org/}}-Library.
Leider mussten wir nach einiger Zeit feststellen, dass diese für unsere Zwecke zu komplex und überladen ist.

Wir haben deshalb eine Komponente erstellt, welche die \brand{Leaflet}\footnote{\url{http://leafletjs.com/}}-Library zur Darstellung der Karte verwendet.
Leaflet ist eine moderne, leichtgewichtige Karten-Library.
Sie wurde speziell für den Einsatz auf mobilen Geräten konzipiert.
Zusätzlich ist sie sehr gut dokumentiert und lässt sich einfach bedienen.

\begin{figure}[H]
	\centering
	\includegraphics[scale=0.5]{images/implementation/frontend/leafletmap-screenshot}
	\caption{Ext.ux.LeafletMap-Komponente in Sencha Touch}
	\label{image-leafletmap-screenshot}
\end{figure}

\subsection{Verfügbarkeit}
Unsere Sencha Touch Komponente war zuletzt so ausgereift, dass wir uns entschieden haben, diese für die Allgemeinheit zugänglich zu machen.
Wir veröffentlichten sie deshalb unter dem Namen \brand{Ext.ux.LeafletMap} im offiziellen Sencha Market\footnote{\url{http://market.sencha.com/}}.
Sie ist verfügbar unter: \url{https://market.sencha.com/users/162/extensions/177}.

\begin{table}[H]
\centering
\begin{tabular}{|p{0.2\twocelltabwidth}|p{0.8\twocelltabwidth}|}
\hline 
\textbf{Ort} & \textbf{URL} \\ 
\hline 
Sencha Market & \url{https://market.sencha.com/users/162/extensions/177} \\ 
\hline 
GitHub & \url{https://github.com/tschortsch/Ext.ux.LeafletMap} \\ 
\hline 
\end{tabular} 
\caption{Ext.ux.LeafletMap Verfügbarkeit}
\label{leafletmap-availiblity}
\end{table}

\begin{figure}[H]
	\centering
	\includegraphics[scale=0.6]{images/implementation/frontend/leafletmap-sencha-market}
	\caption{Leaflet Map-Komponente im Sencha Market}
	\label{image-leafletmap-sencha-market}
\end{figure}

\subsection{Dokumentation}
Die Komponente ist durchgängig mit der Sencha-eigenen JavaScript-Dokumentationssprache \brand{JSDuck}\footnote{\url{https://github.com/senchalabs/jsduck}} dokumentiert. Die Dokumentation befindet sich unter: \url{http://kort.herokuapp.com/docs/Ext.ux.LeafletMap}.